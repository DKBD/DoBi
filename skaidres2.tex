\documentclass[utf8,hyperref={unicode}]{beamer}
%\documentclass[utf8]{beamer}

\mode<presentation>
{
  \usetheme{Berkeley}
  % \usecolortheme{dove} 
   \setbeamertemplate{footline}
   {%
     \leavevmode%
    \hbox{\begin{beamercolorbox}[wd=.5\paperwidth,ht=2.5ex,dp=1.125ex,leftskip=.3cm 	plus1fill,rightskip=.3cm]{author in head/foot}%
   % \usebeamerfont{author in head/foot}\insertshortauthor
    \end{beamercolorbox}%
    \begin{beamercolorbox}[wd=.5\paperwidth,ht=2.5ex,dp=1.125ex,leftskip=.3cm,rightskip=.3cm plus1fil]{title in head/foot}%
    \usebeamerfont{title in head/foot} \hfill p.
	\insertpagenumber\enspace iš \insertdocumentendpage\enspace
    \end{beamercolorbox}}%
  \vskip0pt%
}
}
 \setbeamertemplate{navigation symbols}{}

\usepackage[english]{babel}
%\usepackage[utf8x]{inputenc}
\usepackage[L7x]{fontenc}
\usepackage{lmodern}
\usepackage{amsmath}
\usepackage{amssymb}
%\usepackage{theorem}
\usepackage{bm}
\usepackage{graphicx}
\usepackage{float}
\usepackage{hyperref}

\newcommand{\ab}[1]{#1_{\alpha}}
\newcommand{\wab}[2][\delta]{w_{\alpha}(#2,#1)}
\newcommand{\normab}[1]{\lVert#1\rVert_{\alpha}}
\newcommand{\eps}{\varepsilon}
\newcommand{\sprod}[1]{\langle #1 \rangle}
\DeclareMathOperator{\diam}{diam}
%\renewcommand{\theenumi}{\roman{enumi}}
%\renewcommand{\labelenumi}{\theenumi)}
\theoremstyle{change}\newtheorem{teorema}{Teiginys}
\theoremstyle{change}\newtheorem{salyga}{}
%	\vspace*{20pt}
%	\vspace*{20pt}
\DeclareMathOperator{\seq}{seq}
\DeclareMathOperator{\Var}{Var}
\DeclareMathOperator{\tr}{tr}
\newcommand{\ds}[1]{\displaystyle{#1}}
\newcommand{\dlt}[2]{\Delta^{(#1)}_{#2}}
\newcommand{\norms}[1]{\lVert#1\rVert_{\alpha}^{\seq}}
\newcommand{\normh}[1]{\lVert#1\rVert}
\newcommand{\norma}[1]{\lVert #1\rVert_{\alpha}}

\newcommand{\skirt}[2]{\Delta^{(#1)}_{#2}}
\newcommand{\cp}{\buildrel P\over\longrightarrow}
\renewcommand{\theenumi}{\roman{enumi}}
\newcommand{\R}{\mathbb{R}}
\newcommand{\E}{\mathbf{E}\,} % expectation operator
\newcommand{\bv}{\bm{v}}

\newcommand{\T}{T}
\newcommand{\n}{{\bm{n}}}
\newcommand{\jj}{{\bm{j}}}
\newcommand{\kk}{{\bm{k}}}
\newcommand{\bt}{\bm{t}}
\newcommand{\bu}{\bm{u}}
\newcommand{\B}{\bm{B}}
\newcommand{\N}{\mathbb{N}}
\newcommand{\bi}{\bm{i}}
\newcommand{\p}{\bm{\pi}}
\newcommand{\one}{{\bm{1}}}

\newcommand{\nni}{{\bm{n},\bm{i}}}
\newcommand{\nj}{{\bm{n},\bm{j}}}
\newcommand{\nk}{{\bm{n},\bm{k}}}
\newcommand{\kn}{\bm{k}_{\bm{n}}}

\newcommand{\vv}{\bm{\mathrm{v}}}
\newcommand{\vr}{{\mathrm{v}}}
\newcommand{\uu}{\bm{\mathrm{u}}}
\newcommand{\ur}{{\mathrm{u}}}


\newcommand{\abs}[1]{\left\vert #1 \right\vert}
\newcommand{\snk}{\sigma^2_{\bm{n},\bm{k}}}
\newcommand{\snj}{\sigma^2_{\bm{n},\bm{j}}}

\newcommand{\Rnj}{R_{\bm{n},\bm{j}}}
\newcommand{\Rnk}{R_{\bm{n},\bm{k}}}

\newcommand{\HH}{\mathrm{H}} % a set of notations for Holder spaces
\newcommand{\Ha}{\HH_{\alpha}}
\newcommand{\Hab}{\HH_{\alpha,\beta}}
\newcommand{\Habo}{\HH_{\alpha,\beta}^o}
\newcommand{\Hao}{\HH_{\alpha}^o}
\newcommand{\m}{\mathrm{m}}
\newcommand{\s}{\bm{s}}

\newcommand{\bb}{\bm{\beta}}
\newcommand{\bx}{\mathbf{x}}
\newcommand{\indf}[1]{\mathbf{1}\left( #1 \right)}

\title{Nekilnojamo turto rinkos duomenų analizė}
%\shorttitle{FCRT daugiamačio indekso procesams}
\author[Dovilė Esmantaitė ir Birutė Žabinskaitė]{Dovilė Esmantaitė ir Birutė Žabinskaitė\\ Ekonometrija III kursas\\ Kursinio darbo vadovas: lekt. V. Maniušis}
\institute[Vilnius University] {
    
    Vilniaus Universitetas
    \and
    
    Matematikos ir informatikos fakultetas
 }
\date{2011 gruodžio 8 d.}

\begin{document}
\begin{frame}
    \titlepage
\end{frame}

\begin{frame}
    \frametitle{Tikslai} 
    \begin{itemize} \Large
	\item{ Ištirti, kokie yra namų kainas veikiantys veiksniai;}
            \vskip 20pt
            \item Ištirti, kaip namų kainos priklauso nuo ją veikiančių veiksnių.
    \end{itemize}
\end{frame}

\begin{frame}
    \frametitle{Duomenys} 
    \begin{itemize} \Large
	\item  Tiriami 12 namų kainas veikiantys veiksniai;
	\item Duomenys apima 364 namų aprašus, t.y. 4368 įvairias reikšmes;
	 \item Duomenys grupuojami, norint susiaurinti platų duomenų spektrą.
    \end{itemize}
\end{frame}

\begin{frame}
\frametitle{Duomenų žymėjimas}
\begin{itemize}
\item $Gr$ -- namų grupavimas pagal namo vietą;
\item$Ireng$ -- namų grupavimas pagal jų fizinę būklę;
\item$Kamb$ -- kambarių skaičius; 
\item $Tipas$ -- namas arba kotedžas;
\item$Sodai$ -- namo priklausymas sodų bendrijai; 
\item$Mur$ -- laikančiųjų konstrukcijų rūšis;
\item$Nplotas$ -- namo plotas ($m^2$); 
\item$Splotas$ -- sklypo plotas (a); 
\item$Stat$ -- statybos laikotarpis; 
\item$Baldai$ -- namas su baldais ar ne; 
\item$Atst$ -- namo atstumas iki centro (km).
\end{itemize}
\end{frame}

\begin{frame}\large
    \frametitle{Duomenų grupavimas } 
     \begin{table}[H]
        \begin{center}
        \begin{tabular}{ | p{8cm} | l | l |}
       \hline
         \textbf{Namo vieta} & \textbf{Grupė} \\ \hline
    Senamiestis, Užupis, Žvėrynas & 4 \\ \hline
    Valakampiai, Šnipiškės & 3 \\ \hline
    Antakalnis, Bajorai, Baltupiai, Kalnėnai, Santariškės, Tarandė, Verkiai, Visoriai, Žirmūnai, Zujūnai & 2 \\ \hline
    Aukštieji Paneriai, Balsiai, Dvarčionys, Grigiškės, Naujoji Vilnia, Pagiriai, Pavilnys, Pilaitė, Salininkai, Trakų Vokė, Vilkpėdė, Žemieji Paneriai & 1 \\ \hline
    \end{tabular}
\end{center}
\end{table}
\end{frame}

\begin{frame} 
    \frametitle{Duomenų analizė: pagrindiniai statistiniai rodikliai} 
           \begin{itemize}
	\Large \item Nagrinėjamas vienas kvadratinis metras:
             \end{itemize}
\begin{table}[H]
\begin{center}
 \begin{tabular}{| l | l | l |}
    \hline
    Duomenų skaičius & 364 \\ \hline
    Aritmetinis vidurkis & 3698 Lt \\ \hline
    Standartinis nuokrypis & 2418 Lt \\ \hline
    Minimumas & 1018 Lt\\ \hline
    Maksimumas & 18041 Lt\\ \hline
    Mediana&3146 Lt  \\ 
    \hline
        \end{tabular}
        \end{center}
        \end{table}
\end{frame}

\begin{frame}
    \frametitle{Namo kainos ir vieno kvadratinio metro kainos histogramos} 
    \begin{figure}[H]
  \centering
    \includegraphics[width=0.6\textwidth, height=0.6\textwidth]{4a.pdf}
%\caption{Namų sklaidos diagrama.}
\end{figure}
\end{frame}

\begin{frame}
    \frametitle{Vilniaus miesto rajonų skirstymas į grupes}
    \begin{figure}[H]
  \centering
    \includegraphics[width=0.6\textwidth, height=0.6\textwidth]{Gr4.pdf} 
 % \caption{Vilniaus miesto rajonų skirstymas į grupes.}
\end{figure}
\end{frame}

\begin{frame}
    \frametitle{ Ryšiai tarp namo kainos ir kainą lemiančių veiksnių}
  \begin{table}[H]
%\emph{\caption{Namų grupavimas pagal namo vietą}}
        \begin{center}
        \begin{tabular}{ | p{5,5cm} | c |}
       \hline
       \textbf{Veiksnys} & \textbf{Ryšys} \\ \hline
    Namo vieta & 0,664\\ \hline
    Fizinė būklė & 0,463 \\ \hline
    Atstumas & -0,379 \\ \hline
    Namo tipas & -0,052 \\ \hline
    Kambarių skaičius & 0,320 \\ \hline
    Sodai & -0,261 \\ \hline
    Statyba & 0,045 \\ \hline
    Sklypo plotas & 0,105 \\ \hline
    Buitinė įranga ir baldai & 0,137 \\ \hline
    Namą laikančiųjų konstrukcijų rūšis & 0,199 \\ \hline
    \end{tabular}
\end{center}
\end{table} 
\end{frame}

\begin{frame}
    \frametitle{Pradinis regresinis modelis}   
\begin{align*}
log(Kaina)&=C+\beta_1\cdot Gr2+\beta_2\cdot  Gr3+\beta_3\cdot Gr4\\&+\beta_4\cdot Sodai+\beta_5\cdot Ireng2+\beta_6\cdot Ireng3\\&+\beta_7\cdot Ireng4+\beta_8\cdot Stat+\beta_9\cdot Mur\\&+\beta_{10}\cdot log(Nplotas)+
\beta_{11}\cdot log(Splotas)\\&+\beta_{12}\cdot log(Atstumas)+ \beta_{13}\cdot Baldai\\&+\beta_{14}\cdot Tipas+\beta_{15}\cdot log(Kamb)+e
\end{align*}

\end{frame}



\begin{frame}
   \frametitle{Geriausias regresinis modelis}
\begin{itemize}
         \item Geriausio modelio koeficientų reikšmės:
%\begin{align*}
%log(Kaina)&=9,534+0,234\cdot Gr2+0,618\cdot  Gr3 \\&
%+0,1,029\cdot Gr4-0,188\cdot Sodai\\&+0,205\cdot Ireng2+0,335\cdot Ireng3\\&+0,491\cdot Ireng4+0,355\cdot Statyba\\&-0,183\cdot log(Atstumas)
%\end{align*}
\begin{table}
\begin{tabular}{|c|c|c|c|c|c|c|c|c|c|}
  \hline
\textbf{Kintamasis} & \textbf{Koeficiento reikšmė}\\ \hline
\textbf{Gr2} & 0,234\\ \hline
\textbf{Gr3}& 0,618\\ \hline
\textbf{Gr4}& 1,029\\ \hline
\textbf{Sodai}& -0,188\\ \hline
\textbf{Ireng2}& 0,239\\ \hline
\textbf{Ireng3}& 0,399\\ \hline
\textbf{Ireng4}& 0,674\\ \hline
\textbf{Stat}& 0,321\\ \hline
\textbf{Mur}&0,137\\ \hline
\textbf{log(Nplotas)}& 0,572\\ \hline
\end{tabular}
\end{table}
\end{itemize}
\end{frame}

\begin{frame}
 \frametitle{Geriausio modelio tinkamumas}
\begin{itemize}
           \item Gautasis R$^2$=0,76;
          \item Visų t testų p< 0,05;
          \item Koeficientų ženklai atitinka koreliacijas;
         \item Paklaidos yra normaliosios;
         \item Visi dispersijos mažėjimo daugikliai, t.y. VIF< 4 (nėra multikolinearumo problemos).
    \end{itemize}
 \end{frame}

\begin{frame}
    \frametitle{Išvados}  
    \begin{itemize}
           % \item Namų kainas veikia namo vieta, fizinė būklė, namo tipas, laikančiųjų konstrukcijų rūšis, namo, sklypo plotas, priklausymas sodų bendrijoms, atstumas iki Vilniaus miesto centro.
	\large \item Didžiausią įtaką namo kainai turintys veiksniai: 
                      \begin{itemize}
	            \item  Namo vieta: jei namas patenka į ketvirtą namo vietos grupę, tai vidutinė kaina padidėja 102,9 \%.
                        \item Fizinė būklė: jei namas priklauso ketvirtai fizinės būklės grupės, tai vidutinė kaina padidėja 67,4 \%.	
                       \item Namo plotas: jei namo plotas padidėja 1 \%, tai vidutinė namo kaina padidėja 57,2 \%.
                   \end{itemize}

\end{itemize}
\end{frame}



\end{document}

