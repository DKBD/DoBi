\documentclass[utf8,hyperref={unicode}]{beamer}
%\documentclass[utf8]{beamer}

\mode<presentation>
{
  \usetheme{Berkeley}
  % \usecolortheme{dove} 
   \setbeamertemplate{footline}
   {%
     \leavevmode%
    \hbox{\begin{beamercolorbox}[wd=.5\paperwidth,ht=2.5ex,dp=1.125ex,leftskip=.3cm 	plus1fill,rightskip=.3cm]{author in head/foot}%
    \usebeamerfont{author in head/foot}\insertshortauthor
    \end{beamercolorbox}%
    \begin{beamercolorbox}[wd=.5\paperwidth,ht=2.5ex,dp=1.125ex,leftskip=.3cm,rightskip=.3cm plus1fil]{title in head/foot}%
    \usebeamerfont{title in head/foot}\insertshorttitle \hfill p.
	\insertpagenumber\enspace iš \insertdocumentendpage\enspace
    \end{beamercolorbox}}%
  \vskip0pt%
}
}
 \setbeamertemplate{navigation symbols}{}

\usepackage[english]{babel}
%\usepackage[utf8x]{inputenc}
\usepackage[L7x]{fontenc}
\usepackage{lmodern}
\usepackage{amsmath}
\usepackage{amssymb}
%\usepackage{theorem}
\usepackage{bm}
\usepackage{graphicx}
\usepackage{float}
\usepackage{hyperref}

\newcommand{\ab}[1]{#1_{\alpha}}
\newcommand{\wab}[2][\delta]{w_{\alpha}(#2,#1)}
\newcommand{\normab}[1]{\lVert#1\rVert_{\alpha}}
\newcommand{\eps}{\varepsilon}
\newcommand{\sprod}[1]{\langle #1 \rangle}
\DeclareMathOperator{\diam}{diam}
%\renewcommand{\theenumi}{\roman{enumi}}
%\renewcommand{\labelenumi}{\theenumi)}
\theoremstyle{change}\newtheorem{teorema}{Teiginys}
\theoremstyle{change}\newtheorem{salyga}{}
%	\vspace*{20pt}
%	\vspace*{20pt}
\DeclareMathOperator{\seq}{seq}
\DeclareMathOperator{\Var}{Var}
\DeclareMathOperator{\tr}{tr}
\newcommand{\ds}[1]{\displaystyle{#1}}
\newcommand{\dlt}[2]{\Delta^{(#1)}_{#2}}
\newcommand{\norms}[1]{\lVert#1\rVert_{\alpha}^{\seq}}
\newcommand{\normh}[1]{\lVert#1\rVert}
\newcommand{\norma}[1]{\lVert #1\rVert_{\alpha}}

\newcommand{\skirt}[2]{\Delta^{(#1)}_{#2}}
\newcommand{\cp}{\buildrel P\over\longrightarrow}
\renewcommand{\theenumi}{\roman{enumi}}
\newcommand{\R}{\mathbb{R}}
\newcommand{\E}{\mathbf{E}\,} % expectation operator
\newcommand{\bv}{\bm{v}}

\newcommand{\T}{T}
\newcommand{\n}{{\bm{n}}}
\newcommand{\jj}{{\bm{j}}}
\newcommand{\kk}{{\bm{k}}}
\newcommand{\bt}{\bm{t}}
\newcommand{\bu}{\bm{u}}
\newcommand{\B}{\bm{B}}
\newcommand{\N}{\mathbb{N}}
\newcommand{\bi}{\bm{i}}
\newcommand{\p}{\bm{\pi}}
\newcommand{\one}{{\bm{1}}}

\newcommand{\nni}{{\bm{n},\bm{i}}}
\newcommand{\nj}{{\bm{n},\bm{j}}}
\newcommand{\nk}{{\bm{n},\bm{k}}}
\newcommand{\kn}{\bm{k}_{\bm{n}}}

\newcommand{\vv}{\bm{\mathrm{v}}}
\newcommand{\vr}{{\mathrm{v}}}
\newcommand{\uu}{\bm{\mathrm{u}}}
\newcommand{\ur}{{\mathrm{u}}}


\newcommand{\abs}[1]{\left\vert #1 \right\vert}
\newcommand{\snk}{\sigma^2_{\bm{n},\bm{k}}}
\newcommand{\snj}{\sigma^2_{\bm{n},\bm{j}}}

\newcommand{\Rnj}{R_{\bm{n},\bm{j}}}
\newcommand{\Rnk}{R_{\bm{n},\bm{k}}}

\newcommand{\HH}{\mathrm{H}} % a set of notations for Holder spaces
\newcommand{\Ha}{\HH_{\alpha}}
\newcommand{\Hab}{\HH_{\alpha,\beta}}
\newcommand{\Habo}{\HH_{\alpha,\beta}^o}
\newcommand{\Hao}{\HH_{\alpha}^o}
\newcommand{\m}{\mathrm{m}}
\newcommand{\s}{\bm{s}}

\newcommand{\bb}{\bm{\beta}}
\newcommand{\bx}{\mathbf{x}}
\newcommand{\indf}[1]{\mathbf{1}\left( #1 \right)}

\title{Nekilnojamo turto rinkos duomenų analizė}
%\shorttitle{FCRT daugiamačio indekso procesams}
\author[Dovilė Esmantaitė, Birutė Žabinskaitė]{Dovilė Esmantaitė ir Birutė Žabinskaitė\\ Ekonometrija III kursas\\ Kursinio darbo vadovas: V. Maniušis}
\institute[Vilnius University] {
    
    Vilniaus Universitetas
    \and
    
    Matematikos ir informatikos fakultetas
 }
\date{2011 gruodžio 8 d.}

\begin{document}
\begin{frame}
    \titlepage
\end{frame}

\begin{frame}
    \frametitle{Tikslai} 
    \begin{itemize}
	\item\Large{ Ištirti, kokie yra namų kainas veikiantys veiksniai;}
            \vskip 20pt
            \item Ištirti, kaip namų kainos priklauso nuo ją veikiančių veiksnių.
    \end{itemize}
\end{frame}

\begin{frame}
    \frametitle{Duomenys} 
    \begin{itemize}
	\item\Large Tiriami 12 namų kainas veikiantys veiksniai;
	\item Duomenys apima 390 namų aprašų, t.y. 7800 įvairių reikšmių;
	 \item Duomenys grupuojami.
    \end{itemize}
\end{frame}

\begin{frame}\large
    \frametitle{Duomenų grupavimas } 
     \begin{table}[H]
    %\emph{\caption{Namų grupavimas pagal namo vietą}}
        \begin{center}
        \begin{tabular}{ | p{8cm} | l | l |}
       \hline
       \textbf{Namo vieta} & \textbf{Grupė} \\ \hline
    Senamiestis, Užupis, Valakampiai, Žvėrynas, Šnipiškės & 4 \\ \hline
    Verkiai, Baltupiai, Žirmūnai, Kalnėnai, Santariškės, Visoriai & 3 \\ \hline
    Pilaitė, Bajorai, Pavilnys, Žemieji Paneriai, Grigiškės, Antakalnis, Tarandė, Zujūnai & 2 \\ \hline
    Dvarčionys, Aukštieji Paneriai, Balsiai, Naujoji Vilnia, Pagiriai, Salininkai, Vilkpėdė, Trakų Vokė & 1 \\ \hline
    \end{tabular}
\end{center}
\end{table}
\end{frame}

\begin{frame} 
    \frametitle{Duomenų analizė: pagrindiniai statistiniai rodikliai} 
 	 \begin{table}[H]
%\emph{\caption{Pagrindiniai namų vieno kvadr.m. statistiniai rodikliai }}
             \begin{center}
              \begin{tabular}{| l | l | l |}
    \hline
    Variacijų skaičius & 390 \\ \hline
    Aritmetinis vidurkis & 3905 Lt \\ \hline
    Standartinis nuokrypis & 2418 Lt \\ \hline
    Minimumas & 1018 Lt\\ \hline
    Maksimumas & 18041 Lt\\
    \hline
        \end{tabular}
        \end{center}
        \end{table}
\end{frame}

\begin{frame}
    \frametitle{Namų kainų sklaidos diagrama} 
    \begin{figure}[H]
  \centering
    \includegraphics[width=0.6\textwidth, height=0.6\textwidth]{1.pdf}
%\caption{Namų sklaidos diagrama.}
\end{figure}
\end{frame}

\begin{frame}
    \frametitle{Vilniaus miesto rajonų skirstymas į grupes}
    \begin{figure}[H]
  \centering
    \includegraphics[width=0.6\textwidth, height=0.6\textwidth]{14.pdf} 
 % \caption{Vilniaus miesto rajonų skirstymas į grupes.}
\end{figure}
\end{frame}

\begin{frame}
    \frametitle{ Ryšiai tarp namų vieno kvadratinio metro kainų ir kainas įtakojančių veiksnių}
  \begin{table}[H]
%\emph{\caption{Namų grupavimas pagal namo vietą}}
        \begin{center}
        \begin{tabular}{ | p{5,5cm} | l | l |}
       \hline
       \textbf{Veiksnys} & \textbf{Ryšys} \\ \hline
    Namo vieta & 0,71\\ \hline
    Fizinė būklė & 0,29 \\ \hline
    Atstumas & 0,55 \\ \hline
    Namo tipas & -0,13 \\ \hline
    Kambarių skaičius & 0,03 \\ \hline
    Sodai & 0,18 \\ \hline
    Statyba & 0,05 \\ \hline
    Sklypo plotas & 0,13 \\ \hline
    Buitinė įranga ir baldai & 0,1 \\ \hline
    Namą laikančiosios konstrukcijos & -0,04 \\ \hline
    \end{tabular}
\end{center}
\end{table} 
\end{frame}

\begin{frame}
    \frametitle{Pradinis regresinis modelis}   
\begin{align*}
log(Kvadr.m. kaina)&=C+\beta_1\cdot Gr2+\beta_2\cdot  Gr3+\beta_3\cdot Gr4\\&+\beta_4\cdot Tipas+\beta_6\cdot Sodai+\beta_7\cdot Ireng2\\&+\beta_8\cdot Ireng3+\beta_9\cdot Ireng4\\&+\beta_{10}\cdot Statyba+\beta_{11}\cdot Murinis\\&+
\beta_{12}\cdot Nemuris+\beta_{13}\cdot log(Splotas)\\&+ \beta_{14}\cdot log(Atstumas)+\beta_{15}\cdot Baldai
\end{align*}
\end{frame}



\begin{frame}
   \frametitle{Galutinis regresinis modelis}
\begin{itemize}
           \item Modelis, atspindintis namų vieno kvadratinio metro kainų priklausomybę nuo kainas veikiančių veiksnių:
\begin{align*}
log(Kvadr.m. kaina)&=7,874+0,148\cdot Gr2+0,513\cdot  Gr3 \\&
+0,778\cdot Gr4-0,110\cdot Sodai\\&+0,205\cdot Ireng2+0,335\cdot Ireng3\\&+0,491\cdot Ireng4+0,355\cdot Statyba\\&-0,183\cdot log(Atstumas)
\end{align*}
           \item Gautasis R$^2$=0,62;
          \item Visų t testų p< 0,05;
          \item Koeficientų ženklai atitinka koreliacijas;
         \item Paklaidos yra normaliosios;
         \item Visi VIF< 4 (nėra multikolinearumo problemos).
    \end{itemize}
 \end{frame}

\begin{frame}
    \frametitle{Išvados}  
    \begin{itemize}
           % \item Namų kainas veikia namo vieta, fizinė būklė, namo tipas, laikančiųjų konstrukcijų rūšis, namo, sklypo plotas, priklausymas sodų bendrijoms, atstumas iki Vilniaus miesto centro.
	\item Didžiausią įtaką namų vieno kvadratinio metro kainoms turintys veiksniai: 
                      \begin{itemize}
	            \item  Namo vieta
                        \item Fizinė būklė	
                       \item Statybos laikotarpis
                   \end{itemize}

\end{itemize}
\end{frame}



\end{document}

